\documentclass[a4paper,12pt]{article}
\usepackage{styles/iplouccfg}
\usepackage{styles/zhfontcfg}
\usepackage{styles/iplouclistings}

\graphicspath{{figures/}} 

\title{信号稀疏表示方法的研究进展}
\author{朱亚菲}
\date{2014年10月30日}

\begin{document}

\maketitle

\section{背景}

\subsection{稀疏表示的视觉神经学基础}

人类视觉系统如何对外界信息进行感知与处理一直是脑科学、视觉神经科学及心理学研究的重要问题。目前的研究成果证实,人类视觉系统能够用有限的视觉神经元感知复杂多变的外界输入模式。对此Barlow认为视觉信息神经处理过程的一个重要原则是信息编码的有效性,Olshausen 和Field 进一步提出了稀疏编码原则。稀疏编码的基本原理可以描述为: 对于一个给定的自然图像刺激,仅有一小部分视觉皮层简单细胞被激活;或者说在不同的刺激下,一个特定的视觉皮层简单细胞是很少被激活的。

\subsection{压缩感知理论}
压缩感知理论建立在信号稀疏表示理论的基础上,是一种充分利用信号可压缩性或稀疏性的全新信号获取及处理理论。根据信号采样定理——奈奎斯特定理,只有当采样频率大于信号中最高频率的2倍时,采样之后的数字信号才能完整地保留原始信号中的信息,这样势必会造成资源的巨大浪费。一方面,信号需要通过大于信号带宽两倍的高速率进行采样观测,随着信号带宽的不断增加,对传感系统造成了更大的采样速率方面的压力;另一方面,采样得到的样本太多,对我们存储和传输都带来了巨大的不便,因此我们需要对采样得到的样本数据再进行一定的压缩。这样一来,先以高速率采集得到大量样本然后再压缩就造成了很大的浪费。

于是考虑是不是存在一种采样方式可以直接采样得到适量的信息,并且利用这些信息可以足够好地恢复原始信号?

2004年,Donoho和Candes等人提出了压缩感知理论,它是信号采样时联系模拟信源和数字信息的桥梁,该理论指出:只要信号是可压缩的或在某个变换域是稀疏的,那么就可以利用不相关的观测矩阵直接将这样一个高位信号投影到低维空间上,然后利用少量的投影解一个优化问题,就可以高概率重构原信号。压缩感知理论是一种新的在采样的同时实现压缩的理论框架。

\section{发展}

图像的稀疏表示作为一种图像模型,能够用尽可能简洁的方式表示图像,不仅揭示了图像的内在结构与本质属性,同时能够降低噪声与误差,从而有利于后续的图像处理。

\subsection{基于正交基的稀疏表示}

传统的信号表示方法是基于变换空间展开的,这种基称为“变换基”。如经典的傅里叶变换、DCT变换、小波变换等,都是将信号进行某个正交基空间的投影,表示为$y=\Psi x$。其中,$x$为原始信号,$\Psi$为某个正交基空间,上式其实是将空间域的信号$x$变换到了另外一个空间$\Psi$,在这个空间中信号能得到更好的稀疏表达。但是这种建立在正交基上的信号分解算法也存在一定局限性,因为不同的正交基往往具有不同的特性,如傅里叶变换对振荡信号表达的效果比较好,但对点状奇异性的信号的表示并不有效;而二维小波对图像的点状和板状奇异性表示却非常有效;DCT、Brushlet等变换对图像的纹理特征表达效果更好。不同信号所具有的特点可能相差很大,即便是同一个信号,信号本身的构成也可能错综复杂,单用某一空间正交基对信号进行系数表达,往往不总能得到较好的稀疏表达效果。因此,一些学者开始考虑能否将信号投影到几个不同的组合正交基上,以获得更好的稀疏表达。

\subsection{基于组合正交基的稀疏表示}

2001年Donoho、Candes等人发表的文章即基于组合正交基的思想,文中通过联合小波和曲波变换对图像进行了重构,也是因为考虑到自然现象中的混合信号用单一的正交变换基往往得不到非常有效的表现。2002年Elad、Bruckstein又基于组合正交基提出了不确定性准则和最稀疏解的唯一性结论。基于组合正交基的稀疏分解算法,在构造原子时需要花费一定的时间。但是实验表明,基于组合正交基的稀疏分解方法在总体上计算速度一般能够得到5-10倍的提高。

\subsection{基于多尺度几何分析理论的稀疏表示}

Curvelet变换

Bandelet变换

Contourlet变换

\subsection{基于冗余字典的稀疏表达研究}

最近几年,对稀疏表示研究的另一个热点是信号在冗余字典下的稀疏分解。这是一种全新的信号表示理论:用超完备的冗余函数库取代基函数,称之为冗余字典,字典中的元素被称为原子。字典的选择应尽可能好地符合被逼近信号的结构,其构成可以没有任何限制。从冗余字典中找到具有最佳线性组合的k项原子来表示一个信号,称作信号的稀疏逼近或高度非线性逼近。

当前信号在冗余字典下的稀疏表示研究主要集中在两个方面:

1)如何构造适合信号特点的冗余字典,以及如何提高冗余字典的普适性;

2)用怎样的方法快速设计构造字典,并求得信号的最稀疏表示。

\subsubsection{字典设计方法}

1.从已知的变换基中选取,比如DCT、小波基等,这种方法很通用,但是不能自适应于信号。

2.学习字典,即通过训练和学习大量的与目标数据相似的数据来获得。

字典的学习目前主要有两类:一类是由一组参数和一套选取的含有参数的若干函数构成,用其近似表达信号。这类字典不需要存储整个字典,只需要存储相关参数信息即可,因此大大降低了存储量,但由于涉及的字典与原始信号无关,所以不具有适应性;另一类字典学习方法则是根据信号或图像特点进行训练学习得到的自适应字典。目前用得较多的方法有基于K-均值聚类的思想的K-SVD算法、MOD字典学习算法等。

\subsubsection{字典学习方法}

1.MOD

流程:读入图像(测试图像)$\to$ load预选的字典(一般是训练样本集)$\to$ 使用OMP计算稀疏表示系数

2.K-SVD

流程:读入图像$\to$ load预选的字典(初始值)$\to$ 使用K-SVD生成新的字典与新的表示系数$\to$ 利用新的字典进行下一步研究。






\end{document}
